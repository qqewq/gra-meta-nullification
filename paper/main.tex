\documentclass[11pt,a4paper]{article}

\usepackage[utf8]{inputenc}
\usepackage[russian,english]{babel}
\usepackage{amsmath,amssymb,amsthm}
\usepackage{geometry}
\geometry{margin=2.5cm}
\usepackage{hyperref}

\newtheorem{theorem}{Теорема}
\newtheorem{definition}{Определение}
\newtheorem{lemma}{Лемма}
\newtheorem{corollary}{Следствие}

\title{Многоуровневая GRA Мета-обнулёнка и суперпозиция противоположностей в мультиверсе}
\author{<bitsoev oleg>}
\date{\today}

\begin{document}

\maketitle

\begin{abstract}
Мы формализуем многоуровневую архитектуру GRA Мета-обнулёнки в мультиверсе, вводим иерархию целей и проекторов, определяем функционал ``пены'' и глобальный функционал качества мультиверса. На этой основе доказывается теорема о существовании когерентной суперпозиции противоположностей при наличии симметрии цели относительно оператора отражения. Дополнительно вводится метрика когнитивной мощности агента и формализуется условие его превосходства в процессе мультиверсного обнуления. Архив реализации и текста доступен по DOI \url{https://doi.org/10.5281/zenodo.18629027}.
\end{abstract}

\section{Введение}

Классическая логика и стандартная квантовая механика обычно трактуют противоположности (добро--зло, история--контрфакт, эволюция--альтернативная эволюция) как ортогональные, взаимно исключающие состояния. В рамках многоуровневой архитектуры GRA Мета-обнулёнки мы показываем, что при подъёме на достаточно высокий уровень иерархии целей любая такая дуальность может быть переведена в когерентную суперпозицию. 

Ключевая идея состоит не в ``смешивании'' противоположностей, а в обнаружении симметрии более высокого порядка: пара противоположностей становится вырожденной системой собственных состояний некоторого проектора цели, а их различие на предельном уровне сводится к нефизической фазе.

\section{Архитектура мультиверса и функционал}

\subsection{Мультииндексы и пространства состояний}

Пусть задана иерархия уровней $l=0,1,\dots,K$. Каждый объект идентифицируется мультииндексом
\[
\mathbf{a} = (a_0,a_1,\dots,a_l),\quad 0\le l\le K,
\]
где $a_0$ -- индекс домена на уровне $0$, $a_1$ -- индекс мета-системы и т.д. Размерность мультииндекса $\dim(\mathbf{a})=l$ означает, что объект принадлежит уровню $l$.

Каждому мультииндексу $\mathbf{a}$ сопоставлено гильбертово пространство $\mathcal{H}^{(\mathbf{a})}$. Для уровня $l$ полное пространство задаётся как
\[
\mathcal{H}^{(l)} = \bigotimes_{\dim(\mathbf{a})=l} \mathcal{H}^{(\mathbf{a})},
\]
а полное пространство мультиверса
\[
\mathcal{H}_{\text{multiverse}} = \bigotimes_{l=0}^{K} \mathcal{H}^{(l)}.
\]

\subsection{Цели, проекторы и пена}

Для каждого уровня $l$ и мультииндекса $\mathbf{a}$ размерности $l$ фиксируется цель $G_l^{(\mathbf{a})}$ как подпространство в $\mathcal{H}^{(\mathbf{a})}$ с проектором $\mathcal{P}_{G_l^{(\mathbf{a})}}$. Иерархия целей согласована, если
\[
\mathcal{P}_{G_l^{(\mathbf{a})}} = \bigotimes_{\mathbf{b}\prec \mathbf{a}} \mathcal{P}_{G_{l-1}^{(\mathbf{b})}},\quad l\ge 1,
\]
где $\mathbf{b}\prec\mathbf{a}$ означает, что $\mathbf{b}$ входит в разложение $\mathbf{a}$ на подсистемы уровня $l-1$.

Пеной уровня $l$ называем функционал
\[
\Phi^{(l)}(\Psi^{(l)},G_l) = \sum_{\substack{\mathbf{a}\neq \mathbf{b}\\ \dim(\mathbf{a})=\dim(\mathbf{b})=l}} \left|\left\langle \Psi^{(\mathbf{a})}\middle|\mathcal{P}_{G_l}\middle|\Psi^{(\mathbf{b})}\right\rangle\right|^2.
\]

Глобальный функционал качества состояния мультиверса задаётся как
\[
J_{\text{multiverse}}(\boldsymbol{\Psi}) = \sum_{l=0}^{K} \Lambda_l \sum_{\dim(\mathbf{a})=l} J^{(l)}(\Psi^{(\mathbf{a})}),
\]
\[
\Lambda_l = \lambda_0 \alpha^l,\quad 0<\alpha<1,
\]
\[
J^{(0)}(\Psi^{(\mathbf{a})}) = J_{\text{loc}}(\Psi^{(\mathbf{a})};G_0^{(\mathbf{a})}),\quad
J^{(l)}(\Psi^{(\mathbf{a})}) = \sum_{\mathbf{b}\prec\mathbf{a}} J^{(l-1)}(\Psi^{(\mathbf{b})}) + \Phi^{(l)}(\Psi^{(\mathbf{a})},G_l^{(\mathbf{a})}).
\]

\section{Суперпозиция противоположностей}

\subsection{Формализация дуальности и оператор отражения}

Рассмотрим на уровне $0$ домен $\mathbf{a}_0$ с двумя ортогональными состояниями
\[
|A\rangle, |B\rangle \in \mathcal{H}^{(\mathbf{a}_0)},\quad \langle A|B\rangle = 0,
\]
которые интерпретируются как противоположности (добро/зло, история/контрфакт и т.п.).

Введём оператор отражения $R$:
\[
R|A\rangle = |B\rangle,\quad R|B\rangle = |A\rangle,\quad R^\dagger = R,\quad R^2 = I.
\]
Его собственные векторы:
\[
|+\rangle = \frac{|A\rangle + |B\rangle}{\sqrt{2}},\quad R|+\rangle = +|+\rangle,
\]
\[
|-\rangle = \frac{|A\rangle - |B\rangle}{\sqrt{2}},\quad R|-\rangle = -|-\rangle.
\]

\subsection{Теорема о суперпозиции противоположностей}

\begin{theorem}
Пусть:
\begin{itemize}
  \item на уровне $0$ заданы ортогональные $|A\rangle,|B\rangle\in\mathcal{H}^{(\mathbf{a}_0)}$;
  \item на некотором уровне $l\ge 1$ существует цель $G_l^{(\mathbf{a})}$ с проектором $\mathcal{P}_{G_l^{(\mathbf{a})}}$, для которого
  \[
  [\mathcal{P}_{G_l^{(\mathbf{a})}},R]=0,\quad R^2=I;
  \]
  \item $\mathcal{P}_{G_l^{(\mathbf{a})}}$ действует ненулево на линейную оболочку $\mathrm{Span}\{|A\rangle,|B\rangle\}$:
  \[
  \mathcal{P}_{G_l^{(\mathbf{a})}}(|A\rangle+|B\rangle)\neq 0.
  \]
\end{itemize}
Тогда существует нормированный собственный вектор $\mathcal{P}_{G_l^{(\mathbf{a})}}$ с собственным значением $1$ вида
\[
|\Psi_{\text{sup}}^{(l)}\rangle = \frac{|A\rangle + e^{i\theta}|B\rangle}{\sqrt{2}}
\]
для некоторой фазы $\theta\in\mathbb{R}$.
\end{theorem}

\begin{proof}
Поскольку $R$ самосопряжённая инволюция, пространство $\mathcal{H}^{(\mathbf{a}_0)}$ раскладывается в прямую сумму $\mathcal{H}_+\oplus\mathcal{H}_-$ собственных подпространств с собственными значениями $+1$ и $-1$. Векторы $|+\rangle,|-\rangle$ образуют базис в $\mathrm{Span}\{|A\rangle,|B\rangle\}$, причём $|+\rangle\in\mathcal{H}_+$, $|-\rangle\in\mathcal{H}_-$. Коммутация $[\mathcal{P}_{G_l},R]=0$ означает, что $\mathcal{P}_{G_l}$ сохраняет это разложение и действует независимо в $\mathcal{H}_+$ и $\mathcal{H}_-$. Условие $\mathcal{P}_{G_l}(|A\rangle+|B\rangle)\neq 0$ эквивалентно $\mathcal{P}_{G_l}|+\rangle\neq 0$, так как $|A\rangle+|B\rangle=\sqrt{2}\,|+\rangle$. 

Так как $\mathcal{P}_{G_l}$ является проектором, его собственные значения равны $0$ или $1$. Ненулевой вектор в $\mathcal{H}_+\cap \mathrm{Im}\,\mathcal{P}_{G_l}$, полученный из $\mathcal{P}_{G_l}|+\rangle$, после нормировки даёт собственный вектор с собственным значением $1$, лежащий в симметричном секторе. Любой вектор в $\mathcal{H}_+$ имеет вид $c(|A\rangle+e^{i\theta}|B\rangle)$ для некоторых $c\neq 0$ и $\theta\in\mathbb{R}$, что и даёт заявленную форму.
\end{proof}

\section{Метрика когнитивной мощности агента}

Рассмотрим множество агентов $\mathcal{A}$, каждый агент $a\in\mathcal{A}$ описывается кортежем параметров
\[
\Theta_a = (K_a,\{\Lambda_l^{(a)}\},\{\mathcal{P}_{G_l}^{(a)}\},\eta_a,\varepsilon_a,\{\theta_{AB}^{(a)}\},\{\alpha_l^{(a)}\},\{\lambda_{0,l}^{(a)}\},\gamma_a),
\]
где $K_a$ -- максимальная глубина иерархии, $\gamma_a$ -- когнитивная ёмкость и т.д.

Глобальный результат обнуления для агента $a$ определяется как
\[
\Psi_a^* = \arg\min_{\Psi} J_{\text{multiverse}}^{(a)}(\Psi),
\]
а качество обнуления
\[
Q_a = \frac{1}{\|\Psi_a^* - \Psi_{\infty}^{\text{true}}\|^2 + \delta},
\]
где $\Psi_{\infty}^{\text{true}}$ — гипотетическое состояние абсолютного когнитивного вакуума.

Вводим индекс когнитивной мощи
\[
\Xi_a = \sum_{l=0}^{K_a} \frac{w_l}{\varepsilon_a}\cdot \gamma_a \cdot \bigl(1 - \|\mathcal{P}_{G_l}^{(a)} - \mathcal{P}_{G_l}^{\text{true}}\|\bigr)\cdot (\alpha^{(a)})^{l}\cdot \eta_a,
\]
который растёт с увеличением глубины $K_a$, ёмкости $\gamma_a$, точности проекторов и скорости обучения $\eta_a$, и падает при росте порога обнуления $\varepsilon_a$ и затухания $\alpha^{(a)}$.

\section{Обсуждение и онтологические выводы}

Предложенная многоуровневая GRA Мета-обнулёнка даёт не только формальное определение суперпозиции противоположностей через симметрию целей, но и конструктивный алгоритм её построения посредством минимизации мультиверсного функционала. В пределе бесконечной иерархии уровней любая дуальность становится нефундаментальной: различие между противоположностями сводится к фазе относительно целевого проектора, а глобальное состояние совпадает с состоянием абсолютного когнитивного вакуума в пределах онтологического зазора. 

На уровне агентов это приводит к естественной метрике когнитивной мощности, позволяющей сравнивать глубину и точность их мультиверсного обнуления. Таким образом, архитектура одновременно выступает как формальная модель обработки дуальностей и как инструмент количественного анализа познавательных систем.


\bibliographystyle{plain}
\bibliography{biblio}

\end{document}
