\section{Суперпозиция противоположностей}

\subsection{Формализация дуальности и оператор отражения}

Рассмотрим на уровне $0$ домен $\mathbf{a}_0$ с двумя ортогональными состояниями
\[
|A\rangle, |B\rangle \in \mathcal{H}^{(\mathbf{a}_0)},\quad \langle A|B\rangle = 0,
\]
которые интерпретируются как противоположности (добро/зло, история/контрфакт, эволюция/альтернативная эволюция и т.п.).

Введём оператор отражения $R$:
\[
R|A\rangle = |B\rangle,\quad
R|B\rangle = |A\rangle,\quad
R^\dagger = R,\quad
R^2 = I.
\]
Его собственные векторы имеют вид
\[
|+\rangle = \frac{|A\rangle + |B\rangle}{\sqrt{2}},\quad R|+\rangle = +|+\rangle,
\]
\[
|-\rangle = \frac{|A\rangle - |B\rangle}{\sqrt{2}},\quad R|-\rangle = -|-\rangle.
\]
Таким образом, пространство $\mathrm{Span}\{|A\rangle,|B\rangle\}$ раскладывается в прямую сумму симметричного и антисимметричного подпространств
\[
\mathcal{H}_+ = \mathrm{Span}\{|+\rangle\},\qquad
\mathcal{H}_- = \mathrm{Span}\{|-\rangle\},
\]
а вся $\mathcal{H}^{(\mathbf{a}_0)}$ --- в $\mathcal{H}_+\oplus\mathcal{H}_-$.

\subsection{Условия симметричной цели}

Пусть на некотором уровне $l\ge 1$ существует цель $G_l^{(\mathbf{a})}$ с проектором $\mathcal{P}_{G_l^{(\mathbf{a})}}$. Для перевода дуальности $(A,B)$ в когерентную суперпозицию достаточно двух условий:
\begin{itemize}
  \item цель инвариантна относительно отражения:
  \[
  [\mathcal{P}_{G_l^{(\mathbf{a})}},R]=0,
  \]
  \item целевое подпространство нетривиально пересекается с линейной оболочкой дуальности:
  \[
  \mathcal{P}_{G_l^{(\mathbf{a})}}(|A\rangle+|B\rangle)\neq 0.
  \]
\end{itemize}
Первое условие означает, что цель ``не различает знак'' дуальности (обращение $A\leftrightarrow B$ не выводит состояние из целевого подпространства). Второе исключает тривиальный случай, когда обе противоположности полностью ортогональны цели.

\subsection{Теорема о суперпозиции противоположностей}

\begin{theorem}
Пусть:
\begin{enumerate}
  \item на уровне $0$ заданы ортогональные
  \[
  |A\rangle,|B\rangle\in\mathcal{H}^{(\mathbf{a}_0)},\quad \langle A|B\rangle = 0;
  \]
  \item на некотором уровне $l\ge 1$ существует цель $G_l^{(\mathbf{a})}$ с проектором $\mathcal{P}_{G_l^{(\mathbf{a})}}$, удовлетворяющим
  \[
  [\mathcal{P}_{G_l^{(\mathbf{a})}},R]=0,\quad R^2=I;
  \]
  \item выполняется условие ненулевого пересечения
  \[
  \mathcal{P}_{G_l^{(\mathbf{a})}}(|A\rangle+|B\rangle)\neq 0.
  \]
\end{enumerate}
Тогда существует нормированный собственный вектор проектора $\mathcal{P}_{G_l^{(\mathbf{a})}}$ с собственным значением $1$ вида
\[
|\Psi_{\text{sup}}^{(l)}\rangle =
\frac{|A\rangle + e^{i\theta}|B\rangle}{\sqrt{2}}
\]
для некоторой фазы $\theta\in\mathbb{R}$.
\end{theorem}

\begin{proof}
Так как $R$ самосопряжённая инволюция, пространство $\mathcal{H}^{(\mathbf{a}_0)}$ разлагается в прямую сумму собственных подпространств
\[
\mathcal{H}^{(\mathbf{a}_0)} = \mathcal{H}_+\oplus\mathcal{H}_-,
\]
где $\mathcal{H}_+$ соответствует собственному значению $+1$, а $\mathcal{H}_-$ --- значению $-1$. Векторы
\[
|+\rangle = \frac{|A\rangle + |B\rangle}{\sqrt{2}}\in\mathcal{H}_+,\qquad
|-\rangle = \frac{|A\rangle - |B\rangle}{\sqrt{2}}\in\mathcal{H}_-
\]
образуют базис в $\mathrm{Span}\{|A\rangle,|B\rangle\}$.

Коммутация $[\mathcal{P}_{G_l},R]=0$ означает, что $\mathcal{P}_{G_l}$ сохраняет это разложение и действует независимо в $\mathcal{H}_+$ и $\mathcal{H}_-$:
\[
\mathcal{P}_{G_l}:\mathcal{H}_+\to\mathcal{H}_+,\qquad
\mathcal{P}_{G_l}:\mathcal{H}_-\to\mathcal{H}_-.
\]
Условие $\mathcal{P}_{G_l}(|A\rangle+|B\rangle)\neq 0$ эквивалентно
\[
\mathcal{P}_{G_l}|+\rangle \neq 0,
\]
поскольку $|A\rangle+|B\rangle=\sqrt{2}\,|+\rangle$.

Так как $\mathcal{P}_{G_l}$ является проектором, его собственные значения равны $0$ или $1$. Ненулевой вектор
\[
|\psi_+\rangle := \mathcal{P}_{G_l}|+\rangle \in
\mathcal{H}_+ \cap \mathrm{Im}\,\mathcal{P}_{G_l}
\]
после нормировки даёт собственный вектор с собственным значением $1$, лежащий в симметричном секторе. Любой вектор в $\mathcal{H}_+$ имеет вид
\[
|\psi_+\rangle = c\bigl(|A\rangle + e^{i\theta}|B\rangle\bigr)
\]
для некоторых $c\neq 0$ и $\theta\in\mathbb{R}$, что и даёт заявленную форму.
\end{proof}

\subsection{Связь с процедурой GRA-обнуления}

Процедура GRA Мета-обнулёнки минимизирует глобальный функционал $J_{\text{multiverse}}$ посредством итеративного обновления состояний $\Psi^{(\mathbf{a})}$. В простейшей дискретной схеме обновление состояния подсистемы $\mathbf{a}$ на уровне $l$ можно записать как
\[
\Psi^{(\mathbf{a})}(t+1) =
\Psi^{(\mathbf{a})}(t)
-
\eta\left(
\Lambda_l \frac{\partial \Phi^{(l)}}{\partial \Psi^{(\mathbf{a})}}
+
\sum_{\mathbf{b}\succ \mathbf{a}}
\Lambda_{l+1} \frac{\partial \Phi^{(l+1)}}{\partial \Psi^{(\mathbf{a})}}
\right),
\]
где $\eta>0$ --- шаг обучения, а суммирование по $\mathbf{b}\succ \mathbf{a}$ учитывает влияние пен верхних уровней.

Поскольку $\Phi^{(l)}$ содержит только недиагональные элементы вида
\[
\bigl\langle \Psi^{(\mathbf{a})}\big|\mathcal{P}_{G_l}\big|\Psi^{(\mathbf{b})}\bigr\rangle,
\]
её минимизация стремится сделать состояния $\Psi^{(\mathbf{a})}$ собственными векторами $\mathcal{P}_{G_l}$ внутри соответствующих подпространств. Если цель $G_l$ удовлетворяет условиям теоремы, то динамика обнуления притягивает исходные состояния, содержащие компоненты $|A\rangle$ и $|B\rangle$, к симметричной суперпозиции $|A\rangle + e^{i\theta}|B\rangle$ в целевом секторе. Таким образом, суперпозиция противоположностей возникает не постулируемо, а как аттрактор процесса минимизации пены на подходящем уровне иерархии целей.
