\section{Суперпозиция противоположностей}

\subsection{Формализация дуальности и оператор отражения}

Рассмотрим на уровне $0$ домен $\mathbf{a}_0$ с двумя ортогональными состояниями
\[
|A\rangle, |B\rangle \in \mathcal{H}^{(\mathbf{a}_0)},\quad \langle A|B\rangle = 0,
\]
которые интерпретируются как противоположности (добро/зло, история/контрфакт и т.п.).

Введём оператор отражения $R$:
\[
R|A\rangle = |B\rangle,\quad R|B\rangle = |A\rangle,\quad R^\dagger = R,\quad R^2 = I.
\]
Его собственные векторы:
\[
|+\rangle = \frac{|A\rangle + |B\rangle}{\sqrt{2}},\quad R|+\rangle = +|+\rangle,
\]
\[
|-\rangle = \frac{|A\rangle - |B\rangle}{\sqrt{2}},\quad R|-\rangle = -|-\rangle.
\]

\subsection{Теорема о суперпозиции противоположностей}

\begin{theorem}
Пусть:
\begin{itemize}
  \item на уровне $0$ заданы ортогональные $|A\rangle,|B\rangle\in\mathcal{H}^{(\mathbf{a}_0)}$;
  \item на некотором уровне $l\ge 1$ существует цель $G_l^{(\mathbf{a})}$ с проектором $\mathcal{P}_{G_l^{(\mathbf{a})}}$, для которого
  \[
  [\mathcal{P}_{G_l^{(\mathbf{a})}},R]=0,\quad R^2=I;
  \]
  \item $\mathcal{P}_{G_l^{(\mathbf{a})}}$ действует ненулево на линейную оболочку $\mathrm{Span}\{|A\rangle,|B\rangle\}$:
  \[
  \mathcal{P}_{G_l^{(\mathbf{a})}}(|A\rangle+|B\rangle)\neq 0.
  \]
\end{itemize}
Тогда существует нормированный собственный вектор $\mathcal{P}_{G_l^{(\mathbf{a})}}$ с собственным значением $1$ вида
\[
|\Psi_{\text{sup}}^{(l)}\rangle = \frac{|A\rangle + e^{i\theta}|B\rangle}{\sqrt{2}}
\]
для некоторой фазы $\theta\in\mathbb{R}$.
\end{theorem}

\begin{proof}
Поскольку $R$ самосопряжённая инволюция, пространство $\mathcal{H}^{(\mathbf{a}_0)}$ раскладывается в прямую сумму $\mathcal{H}_+\oplus\mathcal{H}_-$ собственных подпространств с собственными значениями $+1$ и $-1$. Векторы $|+\rangle,|-\rangle$ образуют базис в $\mathrm{Span}\{|A\rangle,|B\rangle\}$, причём $|+\rangle\in\mathcal{H}_+$, $|-\rangle\in\mathcal{H}_-$. Коммутация $[\mathcal{P}_{G_l},R]=0$ означает, что $\mathcal{P}_{G_l}$ сохраняет это разложение и действует независимо в $\mathcal{H}_+$ и $\mathcal{H}_-$. Условие $\mathcal{P}_{G_l}(|A\rangle+|B\rangle)\neq 0$ эквивалентно $\mathcal{P}_{G_l}|+\rangle\neq 0$, так как $|A\rangle+|B\rangle=\sqrt{2}\,|+\rangle$. 

Так как $\mathcal{P}_{G_l}$ является проектором, его собственные значения равны $0$ или $1$. Ненулевой вектор в $\mathcal{H}_+\cap \mathrm{Im}\,\mathcal{P}_{G_l}$, полученный из $\mathcal{P}_{G_l}|+\rangle$, после нормировки даёт собственный вектор с собственным значением $1$, лежащий в симметричном секторе. Любой вектор в $\mathcal{H}_+$ имеет вид $c(|A\rangle+e^{i\theta}|B\rangle)$ для некоторых $c\neq 0$ и $\theta\in\mathbb{R}$, что и даёт заявленную форму.
\end{proof}
