\section{Архитектура мультиверса и функционал}

\subsection{Мультииндексы и пространства состояний}

Пусть задана иерархия уровней $l = 0,1,\dots,K$. Каждый объект идентифицируется мультииндексом
\[
\mathbf{a} = (a_0,a_1,\dots,a_l),\quad 0\le l\le K,
\]
где $a_0$ --- индекс домена на уровне $0$, $a_1$ --- индекс мета-системы, \dots, $a_l$ --- индекс компоненты на уровне $l$. Размерность мультииндекса $\dim(\mathbf{a}) = l$ означает, что объект принадлежит уровню $l$.

Каждому мультииндексу $\mathbf{a}$ сопоставлено гильбертово пространство $\mathcal{H}^{(\mathbf{a})}$. Для уровня $l$ полное пространство задаётся как тензорное произведение всех подсистем этого уровня:
\[
\mathcal{H}^{(l)} = \bigotimes_{\dim(\mathbf{a})=l} \mathcal{H}^{(\mathbf{a})}.
\]
Полное пространство мультиверса:
\[
\mathcal{H}_{\text{multiverse}} = \bigotimes_{l=0}^{K} \mathcal{H}^{(l)}.
\]

\subsection{Цели и проекторы}

Для каждого уровня $l$ и каждого мультииндекса $\mathbf{a}$ размерности $l$ фиксируется цель $G_l^{(\mathbf{a})}$ как подпространство в $\mathcal{H}^{(\mathbf{a})}$ с ортогональным проектором $\mathcal{P}_{G_l^{(\mathbf{a})}}$. Иерархия целей согласована, если
\[
\mathcal{P}_{G_l^{(\mathbf{a})}} = \bigotimes_{\mathbf{b}\prec \mathbf{a}} \mathcal{P}_{G_{l-1}^{(\mathbf{b})}},\quad l\ge 1,
\]
где $\mathbf{b}\prec \mathbf{a}$ означает, что $\mathbf{b}$ входит в разложение $\mathbf{a}$ на подсистемы уровня $l-1$.

Интуитивно: цель верхнего уровня является согласованным объединением целей нижележащих уровней, а проектор цели уровня $l$ действует как тензорное произведение проекторов целей уровня $l-1$ на всех соответствующих подсистемах.

\subsection{Пена уровня и локальные функционалы}

Для состояния $\Psi^{(l)}\in \mathcal{H}^{(l)}$ и цели $G_l$ пена уровня $l$ определяется как
\[
\Phi^{(l)}(\Psi^{(l)},G_l) =
\sum_{\substack{\mathbf{a}\neq \mathbf{b}\\ \dim(\mathbf{a})=\dim(\mathbf{b})=l}}
\left|
\left\langle \Psi^{(\mathbf{a})} \middle| \mathcal{P}_{G_l} \middle| \Psi^{(\mathbf{b})} \right\rangle
\right|^2.
\]
Здесь $\Psi^{(\mathbf{a})}$ и $\Psi^{(\mathbf{b})}$ --- состояния соответствующих подсистем уровня $l$, а $\mathcal{P}_{G_l}$ либо общий проектор цели уровня $l$, либо (в более общем случае) семейство согласованных локальных проекторов.

На уровне $0$ задаётся локальный функционал
\[
J^{(0)}\bigl(\Psi^{(\mathbf{a})}\bigr) =
J_{\text{loc}}\bigl(\Psi^{(\mathbf{a})};G_0^{(\mathbf{a})}\bigr),
\]
отражающий, насколько состояние $\Psi^{(\mathbf{a})}$ соответствует базовой цели уровня $0$ (например, минимизация локальной энергии, правдоподобие данных и т.п.).

Для уровня $l\ge 1$ локальный функционал строится рекурсивно:
\[
J^{(l)}\bigl(\Psi^{(\mathbf{a})}\bigr) =
\sum_{\mathbf{b}\prec \mathbf{a}} J^{(l-1)}\bigl(\Psi^{(\mathbf{b})}\bigr)
\;+\;
\Phi^{(l)}\bigl(\Psi^{(\mathbf{a})},G_l^{(\mathbf{a})}\bigr).
\]
Первое слагаемое аккумулирует вклад всех подсистем уровня $l-1$, второе --- штрафует несогласованность (``пену'') между подсистемами уровня $l$ относительно цели $G_l^{(\mathbf{a})}$.

\subsection{Глобальный функционал мультиверса}

Глобальный функционал качества состояния мультиверса задаётся как
\[
J_{\text{multiverse}}(\boldsymbol{\Psi}) =
\sum_{l=0}^{K} \Lambda_l
\sum_{\dim(\mathbf{a})=l} J^{(l)}\bigl(\Psi^{(\mathbf{a})}\bigr),
\]
где веса уровней
\[
\Lambda_l = \lambda_0\, \alpha^l,\quad 0<\alpha<1,
\]
задают относительную значимость глубинных уровней по сравнению с базовым уровнем $0$.

Минимизация $J_{\text{multiverse}}$ по всем состояниям $\Psi^{(\mathbf{a})}$ реализует процедуру GRA Мета-обнулёнки: пена на каждом уровне стремится к нулю, а состояния подсистем приближаются к собственным векторам соответствующих проекторов целей, согласованных по всей иерархии.
